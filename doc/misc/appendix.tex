%%%%%%%%%%%%%%%%%%%%%%%%%%%%%%%%%%%%%%%%%%%%%%%%%%%%%%%%%%%%%%%%%
%                                                               %
%                       Legal Notice                            %
%                                                               %
% This document is copyright (C) Jason Gobat & Darren Atkinson	%
%                                                               %
%%%%%%%%%%%%%%%%%%%%%%%%%%%%%%%%%%%%%%%%%%%%%%%%%%%%%%%%%%%%%%%%%

\begin{appendix}

\newpage{\pagestyle{empty}\cleardoublepage}

\chapter{Installing and Administering \felt{}}
\label{appendix.install}

\section{Building the \felt{} system from source}

The \felt{} package is intended to be easily portable.  It should build on most 
reasonable Un*x systems without any modifications.  To start the build 
{\tt cd} to the toplevel directory of the \felt{} source tree.  From there 
do a {\tt ./configure}.
This will try to automatically determine where to find relevant include files
and libraries on your system and create a new {\tt etc/Makefile.conf} file.  
If you feel comfortable with it you can go in and tweak 
the {\tt Makefile.conf} file by hand if something does not go right.  
After configuring, do a {\tt make clean} followed by a 
{\tt make all}.  To install the package after it has been successfully 
built do a {\tt make install}.	

The entire package has been compiled and tested under SunOS (4.x and 5.x), 
and Linux (the OS under which it was developed).  This version or earlier 
versions
compiled under HP-UX 8.0 and 9.0 SystemV386 (R3.2.2), and various SGI, DEC, 
and IBM workstations (using Irix, OSF/1 and Ultrix, and AIX) with little or
no problem.  
The files {\em felt.exe} and {\em feltvu.exe} (a graphing application) 
are available as pre-compiled executables for the DOS environment.  	

The most recent version of the source code for \felt{} should always be 
available via anonymous ftp at felt.sourceforge.net in the directory 
{\tt /pub/FElt}.  We also try to make 
pre-compiled binaries available for a wide variety of platforms and operating
systems.  If you want to be made aware of each revision of the \felt{} system 
then we strongly encourage you to subscribe to the \felt{} mailing list.  
To subscribe visit the list information page at 
\mbox{\tt http://lists.sourceforce.net/mailman/listinfo/felt-announce}.
Only major revisions and changes will be announced broadly via Usenet.

\section{Translation files}

The translation files which map non-English terms to the English terms
which \felt{} expects should be installed in the library directory which
you specified during the build.  This path will automatically be
searched when {\em cpp} goes looking for include files (which is really
what the translation files are).  If they are not in some standard place,
users need to specify a {\tt -I} flag on the {\em felt} command line to tell
{\em cpp} where to look for them.  The current filenames are in English,
you'll obviously want to link them or move them to something reasonable
on your system and make users aware of what the appropriate file to include
is called.

The following tables list the currently available translations (that we
have at least).  You can look at the current {\tt german.trn} and 
{\tt spanish.trn} files if you want to change something or provide a new
one.  If you do write a new one, please send it to us.  This is only
a first cut at internationalization, but as we get more translations
we will be in a better position to see how we can make it all work a little
more fluently.  Also, remember that because this is a first cut all you
can really do is type in a \felt{} file by hand using these translations.
Anything {\em velvet} saves or even the file that {\em felt} writes with
the {\tt -debug} command will be in English even if the input file was
given in another language.  All output is still in English.  Lastly, if
you use these translations, you have to use them exactly as they are given
here.  They are not case insensitive, and if there are underscores in the
place of spaces, you must use the underscores.  Things that are user 
specifiable to begin with (object names, problem title) can still be 
whatever you want of course.  If a keyword is not listed in a table below,
use the original English keyword (the English and non-English were probably
equivalent).

{\small
\begin{center}
\begin{tabular}{p{3in}p{3in}} \hline
\bf For German, use \dots	& \bf Instead of \dots \\
\hline 
problembeschreibung 	& problem description \\
titel 			& title \\
knoten 			& nodes \\
elemente 		& elements \\
krafte 			& force \\
kraefte			& forces \\
zwangsbedingung		& constraint \\
zwangsbedingungen 	& constraints \\
materialeigenschaften 	& material properties \\
richtung 		& direction \\
senkrecht 		& perpendicular \\
GlobalesX 		& GlobalX \\
GlobalesY 		& GlobalY \\
GlobalesZ 		& GlobalZ \\
LokalesX 		& LocalX \\
LokalesY 		& LocalY \\
LokalesZ 		& LocalZ \\
werte 			& values \\
verteilte 		& distributed  \\
lasten 			& loads \\
last 			& load \\
gelenk 			& hinge \\
ende 			& end \\
\end{tabular}
\end{center}}

{\small
\begin{center}
\begin{tabular}{p{3in}p{3in}} \hline
\bf For Spanish, use \dots	& \bf Instead of \dots \\
\hline 
descripcion\_del\_problema 	& problem title \\
titulo 				& title \\
nodos 				& nodes \\
elementos 			& elements \\
fuerza 				& force \\
fuerzas				& forces \\
restriccion 			& constraint \\
restricciones 			& constraints \\
propriedades\_del\_material     & material properties \\
direccion 			& direction \\
paralela 			& parallel \\
valores 			& values \\
cargas\_distribuidas 		& distributed loads \\
carga 				& load \\
articulacion\_plana 		& hinge \\
final 				& end \\
\end{tabular}
\end{center}}

\section{Defaults and material databases}

Though the defaults files and material databases are not a necessity for 
{\em velvet}, they are convenient and it might be desirable to install them in 
a location where all users have easy access to them, probably the same
library directory as the translation files.  These files too are really
nothing more than include files which can be included as a convenience for
the user.


%%%%%%%%%%%%%%%%%%%%%%%%%%%%%%%%%%%%%%%%%%%%%%%%%%%%%%%%%%%%%%%%%
% The GNU Public License                                        %
%%%%%%%%%%%%%%%%%%%%%%%%%%%%%%%%%%%%%%%%%%%%%%%%%%%%%%%%%%%%%%%%%

\newpage{\pagestyle{empty}\cleardoublepage}

\chapter{The GNU General Public License}
\label{appendix.gnu-license}

\input{misc/gpl}

\end{appendix}
