%%%%%%%%%%%%%%%%%%%%%%%%%%%%%%%%%%%%%%%%%%%%%%%%%%%%%%%%%%%%%%%%%
%                                                               %
%                       Legal Notice                            %
%                                                               %
% This document is copyright (C) Jason Gobat & Darren Atkinson	%
%                                                               %
%%%%%%%%%%%%%%%%%%%%%%%%%%%%%%%%%%%%%%%%%%%%%%%%%%%%%%%%%%%%%%%%%

\begin{appendix}

\newpage{\pagestyle{empty}\cleardoublepage}

\chapter{Installing and Administering \felt{}}
\label{appendix.install}

\section{Building the \felt{} system from source}

The \felt{} package is intended to be easily portable.  It should build on most 
reasonable Un*x systems without any modifications.  To start the build 
{\tt cd} to the toplevel directory of the \felt{} source tree.  From there 
do a {\tt ./configure}.
This will try to automatically determine where to find relevant include files
and libraries on your system and create a new {\tt etc/Makefile.conf} file.  
If you feel comfortable with it you can go in and tweak 
the {\tt Makefile.conf} file by hand if something does not go right.  
The Geompack library used for element generation is written 
entirely in Fortran but C versions as produced by f2c are included in the 
source distribution so there should be no need for you to f2c them yourself.  
If you have a native Fortran compiler and know how to use it in combination
with C code then you can go ahead and play around with the build process to
make this work. The resulting code will probably be faster than if you 
compile the f2c'd code with a C compiler, but not by much, so unless you 
really care it's probably not worth messing with.  You don't need to have the 
f2c libraries 
to use the converted code and the header file f2c.h is provided in the 
distribution. After configuring, do a {\tt make clean} followed by a 
{\tt make all}.  To install the package after it has been successfully 
built do a {\tt make install}.	

The entire package has been compiled and tested under SunOS (4.x and 5.x), 
and Linux (the OS under which it was developed).  This version or earlier 
versions
compiled under HP-UX 8.0 and 9.0 SystemV386 (R3.2.2), and various SGI, DEC, 
and IBM workstations (using Irix, OSF/1 and Ultrix, and AIX) with little or
no problem.  
The files {\em felt.exe} and {\em feltvu.exe} (a graphing application) 
are available as pre-compiled executables for the DOS environment.  	

The most recent version of the source code for \felt{} should always be 
available via anonymous
ftp at cs.ucsd.edu in the directory {\tt /pub/felt}.  We also try to make 
pre-compiled binaries available for a wide variety of platforms and operating
systems; the currently available set of such binaries and how to get to
them are listed in table~\ref{binary_table}.  
\begin{table}
 \begin{center}
 \begin{tabular}{|l|l|l|}
 \hline
Machine type	& operating system	& location \\
 \hline\hline
i386/i486	& Linux 1.0.x		& ftp://cs.ucsd.edu/pub/felt \\
Sun Sparc	& SunOS 4.1.3		& ftp://cs.ucsd.edu/pub/felt \\
Sun Sparc	& SunOS 5.3 (Solaris 2.3) & ftp://cs.ucsd.edu/pub/felt \\
HP 700 series   & HP-UX 9.05		& \\
SGI Indigo	& Irix 5.x		& \\
IBM RS6000	& AIX x.xx		& \\
DEC 5000	& Ultrix x.xx		& \\
DEC Alpha	& OSF/1 x.xx		& \\
i386/i486	& DOS			& ftp://cs.ucsd.edu/pub/felt \\
 \hline
 \end{tabular}
\end{center}
\caption{Type and location of pre-compiled binaries of the \felt{} system.}
\label{binary_table}
\end{table}

If you want to be
made aware of each revision of the \felt{} system then we strongly encourage
you to subscribe to the \felt{} mailing list.  To subscribe send a one line
email message with ``subscribe felt-l'' (without the quotes) in the body
of the message to listserv@mecheng.asme.org.
Only major revisions and changes will be announced broadly (i.e., via Usenet).

\section{Translation files}

The translation files which map non-English terms to the English terms
which \felt{} expects should be installed in the library directory which
you specified during the build.  This path will automatically be
searched when {\em cpp} goes looking for include files (which is really
what the translation files are).  If they are not in some standard place,
users need to specify a {\tt -I} flag on the {\em felt} command line to tell
{\em cpp} where to look for them.  The current filenames are in English,
you'll obviously want to link them or move them to something reasonable
on your system and make users aware of what the appropriate file to include
is called.

The following tables list the currently available translations (that we
have at least).  You can look at the current {\tt german.trn} and 
{\tt spanish.trn} files if you want to change something or provide a new
one.  If you do write a new one, please send it to us.  This is only
a first cut at internationalization, but as we get more translations
we will be in a better position to see how we can make it all work a little
more fluently.  Also, remember that because this is a first cut all you
can really do is type in a \felt{} file by hand using these translations.
Anything {\em velvet} saves or even the file that {\em felt} writes with
the {\tt -debug} command will be in English even if the input file was
given in another language.  All output is still in English.  Lastly, if
you use these translations, you have to use them exactly as they are given
here.  They are not case insensitive, and if there are underscores in the
place of spaces, you must use the underscores.  Things that are user 
specifiable to begin with (object names, problem title) can still be 
whatever you want of course.  If a keyword is not listed in a table below,
use the original English keyword (the English and non-English were probably
equivalent).

{\small
\begin{center}
\begin{tabular}{p{3in}p{3in}} \hline
\bf For German, use \dots	& \bf Instead of \dots \\
\hline 
problembeschreibung 	& problem description \\
titel 			& title \\
knoten 			& nodes \\
elemente 		& elements \\
krafte 			& force \\
kraefte			& forces \\
zwangsbedingung		& constraint \\
zwangsbedingungen 	& constraints \\
materialeigenschaften 	& material properties \\
richtung 		& direction \\
senkrecht 		& perpendicular \\
GlobalesX 		& GlobalX \\
GlobalesY 		& GlobalY \\
GlobalesZ 		& GlobalZ \\
LokalesX 		& LocalX \\
LokalesY 		& LocalY \\
LokalesZ 		& LocalZ \\
werte 			& values \\
verteilte 		& distributed  \\
lasten 			& loads \\
last 			& load \\
gelenk 			& hinge \\
ende 			& end \\
\end{tabular}
\end{center}}

{\small
\begin{center}
\begin{tabular}{p{3in}p{3in}} \hline
\bf For Spanish, use \dots	& \bf Instead of \dots \\
\hline 
descripcion\_del\_problema 	& problem title \\
titulo 				& title \\
nodos 				& nodes \\
elementos 			& elements \\
fuerza 				& force \\
fuerzas				& forces \\
restriccion 			& constraint \\
restricciones 			& constraints \\
propriedades\_del\_material     & material properties \\
direccion 			& direction \\
paralela 			& parallel \\
valores 			& values \\
cargas\_distribuidas 		& distributed loads \\
carga 				& load \\
articulacion\_plana 		& hinge \\
final 				& end \\
\end{tabular}
\end{center}}

\section{Defaults and material databases}

Though the defaults files and material databases are not a necessity for 
{\em velvet}, they are convenient and it might be desirable to install them in 
a location where all users have easy access to them, probably the same
library directory as the translation files.  These files too are really
nothing more than include files which can be included as a convenience for
the user.

%%%%%%%%%%%%%%%%%%%%%%%%%%%%%%%%%%%%%%%%%%%%%%%%%%%%%%%%%%%%%%%%%%%%%%%%%%%

\newpage{\pagestyle{empty}\cleardoublepage}

\chapter{Geompack Error Codes}
\label{appendix.geompack}

\section{Basic description}

Error codes from 1 to 99 indicate not enough space in some array.
Error codes $>=$ 100 mean that the input specification is incorrect,
there is a program bug, or the tolerance TOL is too small or large
(try a different TOLIN value).  

{\small
\begin{tabular}[h]{cp{5.5in}}
\bf Error Code &	\bf Description of error \\ 
 \hline
   1  &  not enough space in EDGE array for routine EDGHT \\
   2  &  not enough space in HOLV array for routine DSMCPR or DSPGDC \\
   3  &  not enough space in 2-D VCL array \\
   4  &  not enough space in HVL array \\
   5  &  not enough space in PVL, IANG arrays \\
   6  &  not enough space in IWK array \\
   7  &  not enough space in WK array \\
   8  &  not enough space in STACK array for routine SWAPEC, DTRIS2, DTRIW2 \\
   9  &  not enough space in TIL array \\
  10  &  not enough space in CWALK array for routine INTTRI \\
  11  &  not enough space in FC array for 3-D \\
  12  &  not enough space in BF array for 3-D \\
  13  &  not enough space in SVCL, SFVL arrays for routine SHRNK3 \\
  14  &  not enough space in 3-D VCL array \\
  15  &  not enough space in FVL, EANG arrays for polyhedral decomposition \\
  16  &  not enough space in FACEP, NRML arrays \\
  17  &  not enough space in PFL array \\
  18  &  not enough space in HFL array \\
  19  &  not enough space in BTL array \\
  20  &  not enough space in VM array \\
  21  &  not enough space in HT array \\
  22  &  not enough space in FC array for K-D \\
  23  &  not enough space in BF array for K-D \\
 100  &  higher primes need to be added to routine PRIME \\
 200  &  abnormal return in routine DIAM2 \\
 201  &  abnormal return in routine WIDTH2 \\
 202  &  parallel lines in routine SHRNK2 \\
 205  &  horizontal ray to right does not intersect polygon in routine ROTIPG \\
 206  &  out-of-range index when popping points from stack in routine VPRGHT \\
 207  &  out-of-range index when scanning vertices in routine VPSCNA \\
 208  &  out-of-range index when scanning vertices in routine VPSCNB \\
 209  &  out-of-range index when scanning vertices in routine VPSCNC \\
 210  &  out-of-range index when scanning vertices in routine VPSCND \\
 212  &  singular matrix in routine VORNBR \\
 215  &  unmatched separator interface edge determined by routine DSPGDC \\
 216  &  all edges of an outer boundary curve are specified as hole 
	 edges in input to routine DSPGDC \\
 218  &  cannot find subregion above top vertex of hole in routine JNHOLE \\
 219  &  angle at hole vertex in modified region is too far from PI due 
	 to use of relative tolerance in routine JNHOLE \\
 222  &  separator not found in routine MFDEC2 (not likely to occur if 
	 ANGSPC $<=$ 30 degrees and ANGTOL $<=$ 20 degrees) \\
 224  &  2 vertices with identical coordinates (in floating point arithmetic)
	 detected in routine DTRIS2 or DTRIW2 \\
 225  &  all vertices input to DTRIS2 or DTRIW2 are collinear \\
 226  &  cycle detected in walk by routine WALKT2 \\
 230  &  invalid (CW-oriented) triangle created in routine TMERGE \\
 231  &  unsuccessful search in routine FNDTRI 
\end{tabular}}

%%%%%%%%%%%%%%%%%%%%%%%%%%%%%%%%%%%%%%%%%%%%%%%%%%%%%%%%%%%%%%%%%
% The GNU Public License                                        %
%%%%%%%%%%%%%%%%%%%%%%%%%%%%%%%%%%%%%%%%%%%%%%%%%%%%%%%%%%%%%%%%%

\newpage{\pagestyle{empty}\cleardoublepage}

\chapter{The GNU General Public License}
\label{appendix.gnu-license}

\input{misc/gpl}

\end{appendix}
