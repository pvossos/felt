%%%%%%%%%%%%%%%%%%%%%%%%%%%%%%%%%%%%%%%%%%%%%%%%%%%%%%%%%%%%%%%%%
%                                                               %
%                       Legal Notice                            %
%                                                               %
% This document is copyright (C) Jason Gobat & Darren Atkinson	%
%                                                               %
%%%%%%%%%%%%%%%%%%%%%%%%%%%%%%%%%%%%%%%%%%%%%%%%%%%%%%%%%%%%%%%%%

\newpage{\pagestyle{empty}\cleardoublepage}

\prechapter{Foreword}
\label{foreword.intro}

\felt{} is intended as a tool for teaching finite element analysis methods.
There are better tools, there are bigger tools, there are tools that
can do many, many things that \felt{} cannot do.  Those tools are for the
most part not free and if they are, they're usually 20 years old.  
\felt{} is a new, continually evolving system that tries to provide
lots of modern workstation type features and of course, it's completely free.  
 
\presection{About this Manual}

This manual documents {\bf version 3.05} of \felt{}.  It is part of a futile 
attempt to provide comprehensive, accurate documentation for the
\felt{} system.  We do our best to try to keep it up-to-date with the latest 
version of the software; we make no guarantees however, and chances are good
that there are things wrong in here.  If you find something that behaves
differently than the way this document says it should behave then please
let us know.

\presection{Organization of this Manual}

\label{foreword.overview}
This manual is organized in the following way:

{\em Chapter~\ref{intro}} gives you an introduction to the types of problems that
\felt{} can solve.

{\em Chapter~\ref{analysis}} details some of the underlying mathematics for each
of the analysis types supported by \felt{}.

{\em Chapter~\ref{problem}} discusses the basic structure of a \felt{} input file
and how a general finite element problem is translated into the \felt{}
language.

{\em Chapter~\ref{elements}} describes the currently available elements within \felt{}.

{\em Chapter~\ref{felt_prog}} introduces you to the simplest user interface to the \felt{}
system, the command line application {\em felt}.

{\em Chapter~\ref{winfelt}} covers the {\em WinFElt} graphical encapsulator for the 
MS-Windows based environments.

{\em Chapter~\ref{velvet}} introduces {\em velvet}, a full-featured graphical 
environment for the \felt{} system.

{\em Chapter~\ref{velvet.solve}} covers problem solutions and post-processing options within
{\em velvet}

{\em Chapter~\ref{corduroy}} discusses the syntax and usage of {\em corduroy}, the 
command line mesh generator program for \felt{}.

{\em Chapter~\ref{burlap.application}} introduces the powerful and flexible interactive environment 
{\em burlap}.

{\em Chapter~\ref{burlap.syntax}} describes the syntax of {\em burlap} in detail.

{\em Chapter~\ref{algorithm}} describes some of the algorithms that \felt{} uses in solving
an arbitrary problem.

{\em Chapter~\ref{adding_elts}} is an attempt at teaching you how to add elements
to the \felt{} library.

{\em Appendix A} discusses building, installing and administering the \felt{}
system.  A must for potential administrators.

{\em Appendix B} provides a list of Geompack error codes.  You'll want to keep 
this handy if you find yourself doing a lot of mesh generation.

{\em Appendix C} contains a copy of the GNU General Public License, the terms
under which \felt{} is distributed.

\presection{Typographical Conventions}
\label{foreword.convention}

In writing this guide, a number of typographical conventions were employed
to mark buttons, command names, menu options, screen interaction, etc.

\begin{dispitems}
\item [\bf Bold Font]
	Used to mark {\bf buttons}, and {\bf menu options} in graphical
	environments.

\item [\em Italics Font]
	Used to indicate an application program name, e.g. {\em felt}.

\item [\tt Typewriter Font]
	Used to represent screen interaction, either with the {\em velvet} 
        command line, or the shell prompt.
	Also used for example input files, keywords that belong in input
	files and code examples.

\item [\key{Key}]
	Represents a key (or key combination) to press, as in press
        \key{Return} to continue.

\end{dispitems}

\presection{Acknowledgements}
\label{foreword.acknowledgements}

We would like to acknowledge the work of the following people or groups.
Different bits and pieces of their work have either made it possible for us
to develop \felt{} or have contributed to making \felt{} a more functional
and powerful system.

\begin{itemize}
\item
Everyone who has ever worked on the Linux, GNU, X11, and XFree86 projects.
We worked almost exclusively under Linux using gcc as a compiler.  The
X11 project provided a powerful and flexible graphical environment and the
folks at XFree86 made it possible for us to use X11 on our Linux boxes.
\item
Barry Joe developed the Geompack code for triangular mesh generation that
we used in earlier versions of the program.  The new triangular mesh
generator is Triangle by Jonathan Shewchuk.
\item
Some of the ideas for 3d structure plots are based on the way {\it gnuplot}
(by Thomas Williams and Colin Kelley) does it.
\item
The code to generate PostScript graphics files is based on the code
from {\it xmgr} by Paul J. Turner.  The basic look of time-displacement
plots is also based on the way that {\it xmgr} would have drawn them 
because we've always liked the way results from {\it xmgr} looked.
\item
XWD dumps are produced using the same code as in the actual {\it xwd}
application.  The man page says it was authored by Tony Della Fera and
William F. Wyatt.
% \item 
% GIF files are created using code from gifencod.c which was originally 
% written by David Rowley and which I hacked out of {\it ppmtogif} by Jef Poskanzer.  
% The Graphics Interchange Format is copyright CompuServe Inc. and GIF is 
% a service mark of CompuServe Inc.  
\item 
Encapsulated PostScript image files are created using code from {\it pnmtops}
which is part of Jef Poskanzer's fabulous PBMPLUS image format toolkit.
\item
The bivariate interpolation routines are hand translations into C of 
Fortran code originally written by Hiroshi Akima.  The Fortran version
is readily available as one of the ACM-TOMS algorithms.
\item
The routines to do Gibbs-Poole-Stockmeyer/Gibbs-King node renumbering
are also hand translations of Fortran code that was originally published
in ACM-TOMS.
\end{itemize}

\newpage{\pagestyle{empty}\cleardoublepage}
