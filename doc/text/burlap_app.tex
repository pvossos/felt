%%%%%%%%%%%%%%%%%%%%%%%%%%%%%%%%%%%%%%%%%%%%%%%%%%%%%%%%%%%%%%%%%
%                                                               %
%                       Legal Notice                            %
%                                                               %
% This document is copyright (C) Jason Gobat & Darren Atkinson	%
%                                                               %
%%%%%%%%%%%%%%%%%%%%%%%%%%%%%%%%%%%%%%%%%%%%%%%%%%%%%%%%%%%%%%%%%

\newcommand{\adddef}{{\tt add\_definition()}}
\newcommand{\remdef}{{\tt remove\_definition()}}
\newcommand{\rl}{{\tt readline}}

\newpage{\pagestyle{empty}\cleardoublepage}

\chapter{The \burlap\ Application}
\label{burlap.application}

\section{Introduction to the \burlap\ environment}
\label{burlap.intro}

\burlap\ is a mathematical environment designed for adding new element
types to \felt\ and extending the current set of finite element
analyses.  \burlap\ can best be described as a mathematical software
package, similar to {\em matlab} or {\em octave}, but with the data
structures of \felt.  It was designed as an alternative to trying to
understand and modify the C code of the \felt\ library, which was
difficult for even us to do, much less engineers with little C
programming experience.

\burlap\ has its own syntax for manipulating matrices, like that of
{\em matlab}.  However, \burlap\ also includes all of the \felt\ data
types as well, such as nodes, elements, constraints, etc., which
eliminates the need for having large arrays of numbers, where the
third number is the material thickness and the fourth number is the
cross-sectional area, etc.

Any analysis that can be performed with {\em felt} can be performed
with \burlap.  Although the \burlap\ code is slower than the compiled
C code of the \felt\ library, \burlap's interactive environment makes
prototyping new element types and new analyses much easier.  At
present, however, there is no support for executing \burlap\ code from
\velvet.


%%%%%%%%%%%%%%%%%%%%%%%%%%%%%%%%%%%%%%%%%%%%%%%%%%%%%%%%%%%%%%%%%
\section{Using \burlap}
\label{burlap.using}

\subsection{Interacting with \burlap}
\label{burlap.interacting}

The syntax of \burlap\ is designed to be intuitive and is best
illustrated by examples.  A detailed discussion of the syntax can be
found in Chapter~\ref{burlap.syntax}.  The simplest way to start
\burlap\ is to simply type {\tt burlap} on the command line, or {\tt
burlap foo.b} if you wish to execute the \burlap\ file {\tt foo.b}.
You can also specify command line options that qualify what you want
to do.

\begin{dispitems}
\item [\tt -help | -h]
	Display a brief help message which lists all of the available
	command line options and then exits.

\item [\tt -quiet | -q]
	Do not print the start-up message regarding copyright
	information.

\item [\tt -alias | -a]
	Do not define the set of built-in aliases.

\item [\tt -interactive | -i]
	Enter interactive mode by reading expressions from the
	terminal after processing the input files.  Normally, \burlap\
	will simply exit after processing any input files.

\item [\tt -no-interactive | -n]
	Do not enter interactive mode.  This flag tells \burlap\ to
	simply exit if no files are given.  It is useful if you have
	{\tt burlap} aliased to {\tt burlap -i} and wish to override
	the effect of the {\tt -i} flag.

\item [\tt -source command-file | -s command-file]
	Read commands from {\tt command-file} on start-up.

\item [\tt -felt felt-file | -f felt-file]
	Use {\tt felt-file} to define the current \felt\ problem.

\end{dispitems}

Since \burlap\ can load and process a \felt\ file, it also accepts the
standard {\tt -nocpp}, {\tt -cpp}, {\tt -I}, {\tt -U}, and {\tt -D}
flags common to {\em felt} and \velvet.  If you simply type {\tt
burlap}, you will be presented with some copyright information and a
prompt.

\begin{screen}
\begin{verbatim}
This is burlap, copyright 1995 by Jason I. Gobat and Darren C. Atkinson.
This is free software, and you are welcome to redistribute it under certain
conditions, but there is absolutely no warranty.  Type help ("copyright")
for details.  Use the -q option to suppress this message.

[1] _ 
\end{verbatim}
\end{screen}

If \burlap\ was compiled with the GNU \rl\ library, then you have
complete command-line editing and history, as in {\em bash}.  (A
complete discussion of the editing capabilities can be found in the
documentation of the \rl\ library.)  At the prompt, you can enter
expressions to be evaluated.

\begin{screen}
\begin{verbatim}
[1] 1 + 2
[2] write (1 + 2)
3
[3] a = 1 + 2
[4] write (a)
3
[5] write ([1, 2, 3])

         1          2          3 

[6] write ([1; 2; 3])

         1 
         2 
         3 

[7] write ([1, 2, 3] + [4, 5, 6])

         5          7          9 
\end{verbatim}
\end{screen}

As illustrated above, {\tt write()} is used to print results and {\tt
=} is the assignment operator.  Also notice from the first expression
that \burlap\ does not print the result of every expression like {\em
matlab} does.  But, just like {\em matlab}, matrices are delimited by
square brackets with elements separated by commas and rows separated
by semicolons.

If you type {\tt alias} you should see the following list.  (If
\burlap\ is not compiled with the \rl\ library then the list will be
slightly different.)

\begin{screen}
\begin{verbatim}
exit    exit ( )
h       history (20)
help    help ("!$")
ls      system ("ls !*")
quit    exit ( )
\end{verbatim}
\end{screen}

\burlap\ has a very simple syntax with all computations being done
through either operators, such as {\tt +}, or functions, such as {\tt
write()}.  But, since {\tt exit()} and {\tt help()} are rather
non-intuitive, you can alias commands to be whatever you want, just
like {\em csh}.  So, you can just type {\tt exit} or {\tt quit} to
leave \burlap.  Typing {\tt help} will present you with a list of all
the operators and functions that \burlap\ provides.  Typing {\tt help
foo} will give you detailed help on topic {\tt foo}.

In short, \burlap\ works just like most other interactive programs
including {\em csh}, {\em gdb}, {\em gnuplot}, {\em matlab}, {\em
octave}, and {\em gs}.


%%%%%%%%%%%%%%%%%%%%%%%%%%%%%%%%%%%%%%%%%%%%%%%%%%%%%%%%%%%%%%%%%
\section{\burlap\ and \felt}
\label{burlap.felt}

You can load a \felt\ file into \burlap\ by either specifying the file
on the command line with the {\tt -felt} flag or by using the {\tt
felt()} function.

\begin{screen}
\begin{verbatim}
[1] felt ("truss.flt")
\end{verbatim}
\end{screen}

If everything goes okay, then the function should simply return
silently.  Several variables are now automatically defined, as
shown in Table~\ref{burlap.fe.variables}.

\begin{table}[htbp]
\begin{center}
\begin{tabular}{l|l}
Variable	& Description			\\
\hline
\tt nodes	& array of nodes		\\
\tt elements	& array of elements		\\
\tt dofs\_pos	& global \dof\ positions	\\
\tt dofs\_num	& global \dof\ numbers		\\
\tt problem	& problem definition structure	\\
\tt analysis	& analysis parameters structure	\\
\end{tabular}
\caption{Finite element related variables.}
\label{burlap.fe.variables}
\end{center}
\end{table}

\felt\ maintains the global \dofs\ in two vectors.  The first vector,
{\tt dofs\_pos}, is an alias for {\tt problem.dofs\_pos}, and contains
the position number for each of the six \dofs.  A zero entry in the
vector indicates that the \dof\ is not active.  An example vector for
a problem consisting entirely of beam elements would be
%
\begin{displaymath}
\left[ 1\ 2\ 0\ 0\ 0\ 3 \right]
\end{displaymath}
%
since a beam element allows translation along the x-axis and y-axis,
and rotation about the z-axis.  The {\tt dofs\_pos} vector can be used
to determine if a given \dof\ is active and which global number it is
assigned.

The second vector, {\tt dofs\_num}, is an alias for {\tt
problem.dofs\_num}, and contains the numbers of the active \dofs.  The
length of the vector is equal to the number of active \dofs.  The {\tt
dofs\_num} vector corresponding to the {\tt dofs\_pos} vector given
above would then be
%
\begin{displaymath}
\left[ 1\ 2\ 6 \right]
\end{displaymath}
%
since the first, second, and sixth \dofs\ are active.  \burlap\
ensures that the two vectors are always kept consistent.

All arrays start at one, so {\tt nodes (1)} designates the first node
in the problem, and {\tt nodes (1).constraint} designates the
constraint associated with that node.  A period ({\tt .}) is used to
access a field of a \felt\ object.  In general the field names are the
same as the names used in the \felt\ input file.  For example, {\tt
constraint} is the field name for accessing the constraint object
assigned to a node with the {\tt constraint=} syntax in the \felt\
file.

\begin{screen}
\begin{verbatim}
[2] write (nodes)
array of node
[3] write (elements)
array of element
[4] write (nodes (1))
node (1)
[5] write (nodes (1).constraint)
constraint (free)
[6] write (elements (1).material.A)
0.0004
[7] m = elements (1).material
[8] write ("A = ", m.A, " E = ", m.E)
A = 0.0004 E = 2.1e+11
\end{verbatim}
\end{screen}


%%%%%%%%%%%%%%%%%%%%%%%%%%%%%%%%%%%%%%%%%%%%%%%%%%%%%%%%%%%%%%%%%
\subsection{Element objects}

Elements have the same field names as those used in the \felt\ file
with additional fields for holding the computed matrices and stress
values, as shown in Table~\ref{burlap.element.fields}.  For example,
{\tt e.number} will return the element number of the element {\tt e}
and {\tt e.material.t} will return the thickness of the element's
material; {\tt e.nodes (1)} will return the first node of the element
and {\tt e.nodes (1).x} will return the x-coordinate of that node.
Note that {\tt loads} and {\tt distributed} are synonymous, as are
{\tt num\_loads} and {\tt num\_distributed}.

{\scriptsize
\begin{table}[htbp]
\begin{center}
\begin{tabular}{l|l}
Field name		& Description					\\
\hline
\tt number		& element number				\\
\tt nodes		& array of nodes				\\
\tt K			& stiffness matrix				\\
\tt M			& mass matrix					\\
\tt material		& material object				\\
\tt definition		& element type definition			\\
\tt loads		& array of distributed loads			\\
\tt num\_loads		& number of loads				\\
\tt stresses		& array of element stresses			\\
\tt ninteg		& number of integration points			\\
\tt distributed		& array of distributed loads ({\tt loads})	\\
\tt num\_distributed	& number of distributed loads ({\tt num\_loads})\\
\end{tabular}
\caption{Fields of an element object.}
\label{burlap.element.fields}
\end{center}
\end{table}}


%%%%%%%%%%%%%%%%%%%%%%%%%%%%%%%%%%%%%%%%%%%%%%%%%%%%%%%%%%%%%%%%%
\subsection{Node objects}

Like elements, nodes have some fields that are defined through the
\felt\ input file with additional fields to contain the result of
computations, as shown in Table~\ref{burlap.node.fields}.  As an
example, {\tt n.constraint} will return the constraint assigned to the
node {\tt n}, and {\tt n.constraint.name} will return the name of that
constraint; {\tt n.dx} will return the displacement vector and {\tt
n.dx (1)} will return the first component of that vector.  Note that
the displacement vector may be accessed either as a vector by using
{\tt dx}, or as individual components, such as {\tt Tx}.  This is
simply a shorthand; assigning to an individual component is the same
as assigning to the result of indexing the displacement vector.  The
equivalent force vector, {\tt eq\_force}, is used in transforming
distributed loads on an element into equivalent forces on its nodes.

{\scriptsize
\begin{table}[htbp]
\begin{center}
\begin{tabular}{l|l}
Field name	& Description					\\
\hline
\tt number	& node number					\\
\tt constraint	& constraint object				\\
\tt force	& force object					\\
\tt eq\_force	& equivalent force vector			\\
\tt dx		& displacement vector				\\
\tt x		& x-coordinate					\\
\tt y		& y-coordinate					\\
\tt z		& z-coordinate					\\
\tt m		& lumped mass					\\
\tt Tx		& translation along x axis ({\tt dx (1)})	\\
\tt Ty		& translation along y axis ({\tt dx (2)})	\\
\tt Tz		& translation along z axis ({\tt dx (3)})	\\
\tt Rx		& rotation about x axis ({\tt dx (4)})		\\
\tt Ry		& rotation about y axis ({\tt dx (5)})		\\
\tt Rz		& rotation about z axis ({\tt dx (6)})		\\
\end{tabular}
\caption{Fields of a node object.}
\label{burlap.node.fields}
\end{center}
\end{table}}


%%%%%%%%%%%%%%%%%%%%%%%%%%%%%%%%%%%%%%%%%%%%%%%%%%%%%%%%%%%%%%%%%
\subsection{Material objects}

Materials have the fields named by the \felt\ input syntax, as shown
in Table~\ref{burlap.material.fields}.  For example, {\tt m.rho} will
return the density of the material {\tt m} and {\tt m.A} will return
its cross-sectional area.

{\scriptsize
\begin{table}[htbp]
\begin{center}
\begin{tabular}{l|l}
Field name	& Description				\\
\hline
\tt name	& name of material			\\
\tt E		& Young's modulus			\\
\tt Ix		& moment of inertia about x-x axis	\\
\tt Iy		& moment of inertia about y-y axis	\\
\tt Iz		& moment of inertia about z-z axis	\\
\tt A		& cross-sectional area			\\
\tt J		& polar moment of inertia		\\
\tt G		& bulk (shear) modulus			\\
\tt t		& thickness				\\
\tt rho		& density				\\
\tt nu		& Poisson's ratio			\\
\tt kappa	& shear force correction		\\
\tt Rk		& Rayleigh damping coefficient (K)	\\
\tt Rm		& Rayleigh damping coefficient (M)	\\
\tt Kx		& conductivity along x axis		\\
\tt Ky		& conductivity along y axis		\\
\tt Kz		& conductivity along z axis		\\
\tt c		& heat capacity				\\
\end{tabular}
\caption{Fields of a material object.}
\label{burlap.material.fields}
\end{center}
\end{table}}


%%%%%%%%%%%%%%%%%%%%%%%%%%%%%%%%%%%%%%%%%%%%%%%%%%%%%%%%%%%%%%%%%
\subsection{Force objects}

Force objects allow the forces and moments to be accessed as
individual components or as a vector, as shown in
Table~\ref{burlap.force.fields}.  For example, {\tt f.force} will
return the force vector of {\tt f} and {\tt f.force (4)} will return
the moment about the x axis, which is equivalent to {\tt f.force.Mx}.

{\scriptsize
\begin{table}[htbp]
\begin{center}
\begin{tabular}{l|l}
Field name	& Description					\\
\hline
\tt name	& name of force					\\
\tt force	& force vector					\\
\tt spectrum	& input spectra					\\
\tt Fx		& force along x axis ({\tt force (1)})		\\
\tt Fy		& force along y axis ({\tt force (2)})		\\
\tt Fz		& force along z axis ({\tt force (3)})		\\
\tt Mx		& moment about x axis ({\tt force (4)})		\\
\tt My		& moment about y axis ({\tt force (5)})		\\
\tt Mz		& moment about z axis ({\tt force (6)})		\\
\tt Sfx		& spectra along x axis ({\tt spectrum (1)})	\\
\tt Sfy		& spectra along y axis ({\tt spectrum (2)})	\\
\tt Sfz		& spectra along z axis ({\tt spectrum (3)})	\\
\tt Smx		& spectra about x axis ({\tt spectrum (4)})	\\
\tt Smy		& spectra about y axis ({\tt spectrum (5)})	\\
\tt Smz		& spectra about z axis ({\tt spectrum (6)})	\\
\end{tabular}
\caption{Fields of a force object.}
\label{burlap.force.fields}
\end{center}
\end{table}}

For transient problems, a force or moment may vary with time.  In {\em
felt}, this is described by an expression in terms of {\tt t}.  A
component of a force may be assigned a string value which will be
interpreted by \burlap\ as an expression.  Note that evaluating the
component will evaluate the expression at time {\tt t=0}.  There is
currently no way to determine if a component is assigned a
time-varying expression.

\begin{screen}
\begin{verbatim}
[1] f.Fx = "cos (t)"
[2] write (f.Fx)
1
\end{verbatim}
\end{screen}


%%%%%%%%%%%%%%%%%%%%%%%%%%%%%%%%%%%%%%%%%%%%%%%%%%%%%%%%%%%%%%%%%
\subsection{Constraint objects}

Constraint objects also allow the various information to be accessed
as vectors or as individual components, as shown in
Table~\ref{burlap.constraint.fields}.  If {\tt c} is a constraint then
{\tt c.ix} is its initial displacement vector and {\tt c.ix (1)} or
{\tt c.iTx} is its initial translation along the x axis.  Components
may be assigned time-varying expressions, as in the case of forces.

{\scriptsize
\begin{table}[htbp]
\begin{center}
\begin{tabular}{l|l}
Field name	& Description						\\
\hline
\tt name	& name of constraint					\\
\tt constraint	& constraint vector					\\
\tt ix		& initial displacement vector				\\
\tt vx		& initial velocity vector				\\
\tt ax		& initial acceleration vector				\\
\tt dx		& boundary displacement vector				\\
\tt iTx		& initial displacement along x axis ({\tt ix (1)})	\\
\tt iTy		& initial displacement along y axis ({\tt ix (2)})	\\
\tt iTz		& initial displacement along z axis ({\tt ix (3)})	\\
\tt iRx		& initial rotation about x axis ({\tt ix (4)})		\\
\tt iRy		& initial rotation about y axis ({\tt ix (5)})		\\
\tt iRz		& initial rotation about z axis ({\tt ix (6)})		\\
\tt Vx		& initial velocity along x axis ({\tt vx (1)})		\\
\tt Vy		& initial velocity along y axis ({\tt vx (2)})		\\
\tt Vz		& initial velocity along z axis ({\tt vx (3)})		\\
\tt Ax		& initial acceleration along x axis ({\tt ax (1)})	\\
\tt Ay		& initial acceleration along y axis ({\tt ax (2)})	\\
\tt Az		& initial acceleration along z axis ({\tt ax (3)})	\\
\tt Tx		& translation along x axis ({\tt dx (1)})		\\
\tt Ty		& translation along y axis ({\tt dx (2)})		\\
\tt Tz		& translation along z axis ({\tt dx (3)})		\\
\tt Rx		& translation along x axis ({\tt dx (4)})		\\
\tt Ry		& translation along y axis ({\tt dx (5)})		\\
\tt Rz		& translation along z axis ({\tt dx (6)})		\\
\end{tabular}
\caption{Fields of a constraint object.}
\label{burlap.constraint.fields}
\end{center}
\end{table}}


%%%%%%%%%%%%%%%%%%%%%%%%%%%%%%%%%%%%%%%%%%%%%%%%%%%%%%%%%%%%%%%%%
\subsection{Distributed load objects}

Distributed loads, or simply loads, have the fields named by the
\felt\ file syntax, as shown in Table~\ref{burlap.load.fields}.  The
individual values can be accessed using {\tt node} and {\tt
magnitude}.  For example, {\tt l.values (1).node} returns the node
object of the first load point and {\tt l.values (1).magnitude}
returns its magnitude.

{\scriptsize
\begin{table}[htbp]
\begin{center}
\begin{tabular}{l|l}
Field name	& Description						\\
\hline
\tt name	& name of distributed load				\\
\tt direction	& load direction					\\
\tt num\_values	& number of values					\\
\tt values	& array of values (({\tt node}, {\tt magnitude}) pairs)	\\
\end{tabular}
\caption{Fields of a distributed load object.}
\label{burlap.load.fields}
\end{center}
\end{table}}

The {\tt direction} field is used to indicate the direction of the
distributed load.  The direction is simply a scalar value as indicated
in Table~\ref{burlap.load.directions}.

{\scriptsize
\begin{table}[htbp]
\begin{center}
\begin{tabular}{l|l}
Direction		& Value			\\
\hline
\tt LocalX		& \tt \&local\_x	\\
\tt LocalY		& \tt \&local\_y	\\
\tt LocalZ		& \tt \&local\_z	\\
\tt GlobalX		& \tt \&global\_x	\\
\tt GlobalY		& \tt \&global\_y	\\
\tt GlobalZ		& \tt \&global\_z	\\
\tt Parallel		& \tt \&parallel	\\
\tt Perpendicular	& \tt \&perpendicular	\\
\end{tabular}
\caption{Directions of a distributed load.}
\label{burlap.load.directions}
\end{center}
\end{table}}


%%%%%%%%%%%%%%%%%%%%%%%%%%%%%%%%%%%%%%%%%%%%%%%%%%%%%%%%%%%%%%%%%
\subsection{Element definition objects}

An element definition object is one of the few object types not
defined in a \felt\ file.  An element definition, or simply
definition, contains all information necessary for a particular
element type, such as a beam, truss, etc.  The fields given in
Table~\ref{burlap.definition.fields} are discussed at length in
Section~\ref{burlap.adding-elements}.

{\scriptsize
\begin{table}[htbp]
\begin{center}
\begin{tabular}{l|l}
Field name		& Description					\\
\hline
\tt name		& name of definition				\\
\tt setup		& element set-up function			\\
\tt stress		& element stress computation function		\\
\tt num\_nodes		& number of nodes in element			\\
\tt shape\_nodes	& number of nodes that define shape		\\
\tt num\_dofs		& number of degrees of freedom			\\
\tt dofs		& degrees of freedom				\\
\tt num\_stresses	& number of stresses				\\
\tt retainK		& flag to retain stiffness matrix after assembly\\
\end{tabular}
\caption{Fields of an element definition object.}
\label{burlap.definition.fields}
\end{center}
\end{table}}


%%%%%%%%%%%%%%%%%%%%%%%%%%%%%%%%%%%%%%%%%%%%%%%%%%%%%%%%%%%%%%%%%
\subsection{Problem definition}

The problem definition object contains all information about the
current \felt\ problem, with the exception of the analysis parameters.

{\scriptsize
\begin{table}[hpbt]
\begin{center}
\begin{tabular}{l|l}
Field name		& Description			\\
\hline
\tt mode		& analysis mode			\\
\tt title		& problem title			\\
\tt nodes		& array of node objects		\\
\tt elements		& array of element object	\\
\tt dofs\_pos		& global \dof\ positions	\\
\tt dofs\_num		& global \dof\ numbers		\\
\tt num\_dofs		& number of active \dofs	\\
\tt num\_nodes		& number of nodes		\\
\tt num\_elements	& number of elements		\\
\end{tabular}
\caption{Fields of a problem definition object.}
\label{burlap.problem.fields}
\end{center}
\end{table}}

Most fields of the problem definition given in
Table~\ref{burlap.problem.fields} are read-only.  If you want to
change the characteristics of a problem, you need to edit the input
file that defines the problem, and load it with the {\tt felt()}
function.  The only exceptions are the \dof\ arrays, {\tt dofs\_pos}
and {\tt dofs\_num}.  Since these arrays must be properly initialized
before many of the built-in finite element functions may be called,
\burlap\ ensures that the arrays are kept consistent.  The
individual components of the arrays are read-only; however, either
array may be assigned a row vector to change the set of active \dofs.
When one array is changed, both arrays and the number of active \dofs\
are updated.  (We don't know why you would want to do this.  Usually,
{\tt find\_dofs()} should be used to initialize the arrays.  Changing
the \dofs\ yourself will probably just lead to non-singular matrices.)


%%%%%%%%%%%%%%%%%%%%%%%%%%%%%%%%%%%%%%%%%%%%%%%%%%%%%%%%%%%%%%%%%
\subsection{Analysis parameters}

The analysis parameters structure contains information related to
a specific analysis type.

{\scriptsize
\begin{table}[htbp]
\begin{center}
\begin{tabular}{l|l}
Field name	& Description				\\
\hline
\tt gamma	& parameter in Newmark integration	\\
\tt beta	& parameter in Newmark integration	\\
\tt alpha	& parameter in H-H-T			\\
\tt mass\_mode	& mass mode				\\
\tt nodes	& array of node numbers of interest	\\
\tt dofs	& array of dofs of interest		\\
\tt start	& starting time or frequency		\\
\tt stop	& ending time or frequency		\\
\tt step	& time or frequency increment		\\
\tt num\_dofs	& number of dofs of interest		\\
\tt num\_nodes	& number of nodes of interest		\\
\end{tabular}
\caption{Fields of an analysis parameters object.}
\label{burlap.analysis.fields}
\end{center}
\end{table}}

Unlike the problem definition structure, most of the fields in the
analysis parameters structure are changeable.  However, the {\tt dofs}
and {\tt nodes} fields are similar to the {\tt dofs\_num} and {\tt
dofs\_pos} fields of the problem definition.  The individual
components are read-only, but they may be assigned a row vector.  If
{\tt analysis.dofs} is assigned a valid row vector then {\tt
analysis.num\_dofs} is updated and is equivalent to {\tt length
(analysis.dofs)}.  Similarly, if {\tt analysis.nodes} is assigned a
valid row vector then {\tt analysis.num\_nodes} is updated and is
equivalent to {\tt length (analysis.nodes)}.  The {\tt length()}
function returns the number of items in an array, matrix, or string.


%%%%%%%%%%%%%%%%%%%%%%%%%%%%%%%%%%%%%%%%%%%%%%%%%%%%%%%%%%%%%%%%%
\section{Adding new element types to \burlap}
\label{burlap.adding-elements}

Adding a new element type to \felt\ can be very time-consuming and
error-prone.  \burlap\ allows the user to easily add new element
definitions that can be quickly tested without having to recompile the
{\em felt} or \velvet\ applications.  To add a new element to the set
of existing elements, the \adddef\ function is used.  Similarly, to
remove an existing element definition, the \remdef\ function is used.

\begin{table}[htbp]
\begin{center}
\begin{tabular}{l|l|l|l}
Argument	  & Description			      & Type	       & Example	    \\
\hline
\tt name	  & name of definition		      & \tt string     & \tt "truss"	    \\
\tt set\_up	  & set up function ($K$ and $M$)     & \tt function   & \tt truss\_set\_up \\
\tt stress	  & stress computation function	      & \tt function   & \tt truss\_stress  \\
\tt shape	  & element shape		      & \tt scalar     & \tt \&linear	    \\
\tt num\_nodes	  & number of nodes in element	      & \tt scalar     & \tt 2		    \\
\tt shape\_nodes  & number of nodes that define shape & \tt scalar     & \tt 2		    \\
\tt num\_stresses & number of stress values	      & \tt scalar     & \tt 1		    \\
\tt dofs	  & vector of \dofs		      & \tt row vector & \tt [1,2,3,0,0,0]  \\
\tt retainK	  & retain stiffness matrix indicator & \tt scalar     & \tt \&false	    \\
\end{tabular}
\caption{Arguments to the \adddef\ function.}
\label{burlap.adddef.arguments}
\end{center}
\end{table}

The \adddef\ function requires a variety of arguments as shown in
Table~\ref{burlap.adddef.arguments}.  For the reminder of the
discussion regarding the arguments to the \adddef\ function, we will
use the truss element as an example.

The {\tt name} of the element definition is a string by which the
element type is referred.  Once the element definition has been
successfully added then new elements of that type can be created.  In
our example the name is {\tt "truss"}, so a \felt\ file can now be
loaded that contains {\tt truss} elements.  If an element definition
of the same name already exists then \adddef\ returns {\tt 1},
otherwise it returns {\tt 0}.

The {\tt set\_up} function is a \burlap\ function that will compute
the local stiffness matrix, $K$, and local mass matrix, $M$, for any
element of the new definition.  Our sample set-up function is shown in
Figure~\ref{burlap.truss.setup}.  For our simple truss element, we are
ignoring the mass matrix and the effect of distributed loads.

\begin{figure}[htbp]
\begin{center}
\begin{verbatim}
function truss_set_up (e)
    L = length (e)
    AEonL = e.material.A * e.material.E / e.material.L
    K = [AEonL, -AEonL; -AEonL, AEonL]

    cx = (e.node (2).x - e.node (1).x) / L
    cy = (e.node (2).y - e.node (1).y) / L
    cz = (e.node (2).z - e.node (1).z) / L

    T = [cx, cy, cz, 0, 0, 0; 0, 0, 0, cx, cy, cz]
    e.K = T'*K*T

    return 0
end
\end{verbatim}
\caption{Set-up function for the truss element definition.}
\label{burlap.truss.setup}
\end{center}
\end{figure}

The set-up function will be called with two arguments.  The first
argument is the element whose stiffness and mass matrices are to be
computed.  The second argument is the mass mode, which is one of {\tt
\&false}, {\tt \&lumped}, or {\tt \&consistent}, where {\tt \&false}
indicates that no mass matrix is required.  In our simple example, we
have simply omitted the second argument, since we will not be
considering mass matrices.  Note that the {\tt set\_up} argument is a
\burlap\ function itself, not the name of the function.  As discussed
in Section~\ref{burlap.syntax.functions}, functions in \burlap\ are
simply variables that can be passed as arguments to functions and also
returned from functions.  Our simple function for truss elements
computes the length of the element by calling the {\tt length()}
function, which is discussed in Section~\ref{burlap.fe.functions}.

The set-up function assigns the stiffness and mass matrices to the
element by assigning to the {\tt K} and {\tt M} fields of the element
structure.  If the set-up function is successful then it should return
zero.  Otherwise, it should return a non-zero value.

The {\tt stress} function is a \burlap\ function that will compute the
stress values for any element of the new definition.  Our sample
set-up function is shown in Figure~\ref{burlap.truss.stress}.  The
stress function assigns the stresses by assigning to the {\tt stress}
field of the element structure.  If the stress function is successful
then it should return zero.  Otherwise, it should return a non-zero
value.

\begin{figure}[htbp]
\begin{center}
\begin{verbatim}
function truss_stress (e)
    L = length (e)
    EonL = e.material.E / L
    c = zeros (6, 1)

    c(1) = (e.node (2).x - e.node (1).x) / L
    c(2) = (e.node (2).y - e.node (1).y) / L
    c(3) = (e.node (2).z - e.node (1).z) / L

    e.ninteg = 1
    e.stress (1).x = (e.node(1).x + e.node(2).x) / 2
    e.stress (1).y = (e.node(1).y + e.node(2).y) / 2
    e.stress (1).values (1) = EonL * (e.node(2).dx - e.node(1).dx) * c

    return 0
end
\end{verbatim}
\caption{Stress function for the truss element definition.}
\label{burlap.truss.stress}
\end{center}
\end{figure}


Our simple truss element has only one integration point and only a
single stress value at that point.  Since an element may have an
arbitrary number of integration points, the {\tt ninteg} field of the
element structure must be set before the {\tt stress} vector can be
accessed.  The number of stresses and thus the length of the {\tt
values} vector is determined by the element definition.  In the
computation of the stress value for the truss element, the two
displacement row vectors are subtracted and then multiplied by the
column vector, {\tt c}, to yield a scalar.

The remaining arguments to \adddef\ are easily calculated.  The shape
of our truss element is linear and has two nodes, both of which are
required and needed to define the shape of the element.  The truss
element permits displacements along the X, Y, and Z axes, but does not
allow for rotation about any of the axes.  Therefore, the first,
second, and third \dofs\ are active.  The entire call to \adddef\ is:

\begin{center}
{\tt add\_definition ("truss", truss\_set\_up, truss\_stress, \\
\&linear, 2, 2, 1, [1, 2, 3, 0, 0, 0])}
\end{center}

We have neglected to pass a value for the last argument, which
indicates whether the local stiffness matrix should be retained after
assembly into the global stiffness matrix.  The default value for this
argument is {\tt \&false}, which is fine for our purposes.  Only the
last argument to \adddef\ may be omitted; the remaining argument do
not have default values.  If we wish to remove the truss element
definition, then the \remdef\ function must be used.  The \remdef\
function takes a single argument that is the name of the element to be
removed.  In our case, the call would be {\tt remove\_definition
("truss")}.


%%%%%%%%%%%%%%%%%%%%%%%%%%%%%%%%%%%%%%%%%%%%%%%%%%%%%%%%%%%%%%%%%
\section{Tips on using interactive mode}
\label{burlap.interactive-tips}

Most of the time, \burlap\ will be used in interactive mode:
expressions will be typed in at the prompt and evaluated.  Functions
are best created with an editor as a file which is then included.

\begin{screen}
\begin{verbatim}
[1] system ("vi foo.b")
[2] include ("foo.b")
[3] system ("vi foo.b")
[4] include ("foo.b")
\end{verbatim}
\end{screen}

It soon becomes tedious to retype the same expressions.  If \burlap\
is compiled with the \rl\ library, then you can use the built-in history
mechanism.

\begin{screen}
\begin{verbatim}
[1] system ("vi foo.b")
[2] include ("foo.b")
[3] !s
[4] !i
\end{verbatim}
\end{screen}

For those not familiar with {\tt csh} or {\tt bash}, {\tt !s} executes
the last line that begins with {\tt s}.  Similarly, {\tt !i} executes
the last line that begins with {\tt i}.  You can save yourself more
typing by using the word completion abilities of the \rl\ library.
Pressing the tab key will complete the current word.  If there is not
a unique completion, then you will hear a bell and pressing the tab
key again will list the completions.

\begin{screen}
\begin{verbatim}
[1] sy <Tab>
symmetric? system
[1] system ("vi foo.b")
[2] in <Tab>
in                    integrate_hyperbolic  inv
include               integrate_parabolic   
[2] include ("foo.b")
\end{verbatim}
\end{screen}

The default set of built-in completions includes all keywords such as
{\tt function} and {\tt return}, intrinsic functions such as {\tt
include} and {\tt sin}, and enumeration constants such as {\tt
\&linear} and {\tt \&null}.  Inside a quoted string, the default set
of completions is not used; instead, file names are completed.

\begin{screen}
\begin{verbatim}
[1] include ("Examples/f <Tab>
fe.b         find-dofs.b  foo.flt      foo_d.flt    
[1] include ("Examples/f
\end{verbatim}
\end{screen}

Using the completion mechanism can save a lot of typing, especially
when referring to file names.  Another trick is to define your own
aliases for common lines.

\begin{screen}
\begin{verbatim}
[1] alias v system ("vi foo.b")
[2] alias vi system ("vi !$")
[3] alias print write (!*)
[4] vi foo.b
\end{verbatim}
\end{screen}

The first alias defines {\tt v} to be {\tt system ("vi foo.b")}.  The
second is more general and defines {\tt vi} to be {\tt system ("vi
!\$")}.  The string {\tt !\$} will be replaced with the last word on
the line.  A complete description of the history expansion can be
found in the documentation for the \rl library and for {\tt csh}.  The
final alias, {\tt print}, is simply a convenient wrapper for the {\tt
write} function.

If \burlap\ is not compiled with the \rl\ library, you still have
access to a limited form of aliasing.  The only expansion string legal
in an alias is {\tt \%s} which will expand to the entire line except
the first word.  A similar alias to {\tt vi} above would be {\tt
system ("vi \%s")}.

An alternative to writing aliases is to define your own convenience
functions.  If you wish to be able to both edit and include a file
then you can write a small function, called {\tt edit}, and place it
in a file called {\tt edit.b} in your search path, as described in
Section~\ref{burlap.syntax.function-call}.

\begin{screen}
\begin{verbatim}
function edit (s)
    system (concat ("vi ", s))
    include (s)
end
\end{verbatim}
\end{screen}

Finally, it is important to realize that the \rl\ library capabilities
and aliases work on lines, not on expressions.  History substitutions
apply to the lines entered, not to the individual expressions on a
line.  Similarly, aliases and history expansion characters are not
recognized when \burlap\ is reading from a file.


%%%%%%%%%%%%%%%%%%%%%%%%%%%%%%%%%%%%%%%%%%%%%%%%%%%%%%%%%%%%%%%%%
\section{Common error messages}
\label{burlap.errors}

This section attempts to explain some of the common error messages
associated with using \burlap.  All error messages are prefixed with
the name of the file and the number of the line that was found to be
in error.  An error on a line may not be discovered until several
lines later, so the line number may not be the line that actually
contained the error.  If an error occurs in a function, then the
function call is terminated and the error is propagated back to the
outermost level.  In this case, the calling sequence is shown to
assist in tracking down the cause of the error.  In the examples
given, the file name and line number have been omitted.

\begin{dispitems}

\item[\tt type error in expression: function call to null]
This error indicates that the function call or array index operators
were applied to a {\tt null} expression such as {\tt foo ( )}, where
{\tt foo} is {\tt null}.  This error often results from misspelling a
variable name or from forgetting to declare a global variable using
the {\tt global} declaration in a function body.  You can generally
locate this error by searching for parentheses on the offending line.

\item[\tt type error in expression: \it type \tt has no such field]
The field index operator was applied to an object of the specified
{\it type}, but the object has no such field.  You can generally
locate this error by searching for a period on the given line.

\item[\tt type error in expression: \it type1 operator type2]
The specified {\it operator} does not allow operands of the specified
types.  The way to fix this error is to determine what operand types
are allowed, as discussed in Section~\ref{burlap.syntax.operators},
and determine why one or more of your operands are of the incorrect
type.

\item[\tt parse error before \it string]
At some time at or before {\it string}, the expression syntax was
illegal.  Often, {\it string} is the cause of the error.  You should
now be back at the top level prompt, in which case you can correct
your error and try again.

\item[\tt size mismatch in expression: ({\it a} x {\it b}) {\it operator}
({\it c} x {\it d})]
The types of the operands were legal for the specified {\it operator},
but the sizes of the operands were illegal.  The legal operand sizes
are discussed in Section~\ref{burlap.syntax.operators}.

\end{dispitems}
