%%%%%%%%%%%%%%%%%%%%%%%%%%%%%%%%%%%%%%%%%%%%%%%%%%%%%%%%%%%%%%%%%
%                                                               %
%                       Legal Notice                            %
%                                                               %
% This document is copyright (C) Jason Gobat & Darren Atkinson	%
%                                                               %
%%%%%%%%%%%%%%%%%%%%%%%%%%%%%%%%%%%%%%%%%%%%%%%%%%%%%%%%%%%%%%%%%

\newpage{\pagestyle{empty}\cleardoublepage}

\chapter{The {\em felt} Application}
\label{felt_prog}

\section{Using {\em felt}}
\label{felt_prog.using}

The basic way to solve a problem with {\em felt} is simply to type 
{\tt felt foo.flt} on the command line, where {\tt foo.flt} is the name of a 
\felt{} input file.  There are several command line options, however, which 
offer additional capabilities.

\begin{dispitems}
\item [\tt -version]
        display the current version number of {\em felt} and exit.

\item [\tt -help]
        display a brief help message which lists all of the available command
        line options.

\item [\tt -debug]
	will generate a \felt{} file, that if all is working well, 
	should look exactly like the original input file.  The 
	generated file represents what the application thinks it was 
	given.

\item [\tt -preview]
        produces an ASCII rendering of the problem geometry.  The graphics
        may not be great but the result is often good enough for a simple
        sanity check.

\item [\tt -renumber]
        will invoke a Gibbs-Poole-Stockmeyer/Gibbs-King automatic node
        renumbering algorithm for bandwidth/profile reduction of the 
        global stiffness matrix (and mass and damping matrices in 
        transient analysis).  The renumbering will only affect solution
        time and memory requirements; results will be referenced to
        the originally assigned node numbers.  Generally there is no
        benefit in using this capability for problems with small numbers
        of nodes.

\item [\tt +table]
        disables tabular output for transient and spectral analysis
        problems. 

\item [\tt -plot]
        generates an ASCII character based plot (or plots) of results
        for transient or spectral analysis problems.

\item [\tt -transfer]
        only calculate the transfer functions during spectral analysis.  The
        output can be graphical and/or tabular depending on the {\tt plot}
        and {\tt table} options.

\item [\tt -eigen]
        in a modal analysis problem, only calculate the eigenvalues
        (natural frequencies) and eigenvectors (mode shapes) before quitting
        (i.e., do not calculate modal mass, damping and stiffness matrices).

\item [\tt -orthonormal]
        use orthonormal mode shapes to calculate modal mass, damping, and
        stiffness matrices in a modal analysis problem. 

\item [\tt -matrices]
	will print the global matrices that are appropriate to this problem.
        In static analysis this would simply be the stiffness matrix;
        in transient analysis this is the stiffness, damping, and 
        mass matrices; in modal analysis, this is the condensed stiffness,
        damping, and mass matrices.

\item [\tt -summary]
	will generate a summary of all the material properties that 
	were used in the problem.  The number of elements using a material,
	total length, surface area and mass of that material (if 
	masses are desired, then the material properties definition 
	must include a density) and the total mass of the structure
        are given.

\item [\tt -graphics foo1]
	will create a graphics file called foo1 so the user can 
	visualize the structure with a standard graphing package to 
	make sure that everything is connected where it should be.  
	The format of this file will be a list of coordinate triplets 
	connecting each element, with blank lines delimiting the end 
	of an element.

\item [\tt -cpp filename]
	substitute {\tt filename} for the pre-processor to run on the 
	input file.

\item [\tt -nocpp]
   	do not run the input file through the pre-processor.

\item [\tt -Idirectory]
	add {\tt directory} to the standard search path for include files in
	the pre-processor.

\item [\tt -Uname]
   	undefine the macro {\tt name} in the pre-processor.

\item [\tt -Dname=value]
	define {\tt name} to be the macro {\tt value} in the pre-processor.
\end{dispitems}

Additionally, every \felt{} file that is run through {\em felt} is 
pre-processed by the C preprocessor (except in DOS).  This allows for the use 
of macro definitions and include files within a \felt{} input file.  An example
of this capability is the international translation files which allow you
to specify a \felt{} input file in a language other than English.  These
translation files are nothing more than include files which contain 
{\tt \#define}
statements which map the non-English terms to what \felt{} actually expects 
(the regular English terms).  An input file that takes advantage of these
powerful features might look like this:
\begin{screen}
 \begin{verbatim}
#include "german.trn"
#define map(x) ((x)*(cos((x))*cos((x)) + sin((x))*sin((x))))

problembeschreibung
titel="A cantilever beam" knoten=3 elemente=2 

knoten
1 x=0        y=0    krafte=point_load      zwangsbedingung=pin
2 x=map(240) y=0                           

beam elemente
1 knoten=[1,2] material=steel

materialeigenschaften
steel e=30e6 ix=100 a=10

kraefte
point_load fy=-1000

zwangsbedingungen
pin tx=c ty=c rz=u

ende
 \end{verbatim}
\end{screen}
The C preprocessor options and 
capabilities are described more fully in the {\em felt}(1fe) manual page.
The international translation capability is
discussed more fully (including current translation tables) 
in appendix~\ref{appendix.install}.

\section{Solving a problem}
\label{felt_prog.solve}

As mentioned above the easiest way to solve a problem is to simply do 
\begin{screen}
 \begin{verbatim}
% felt foo.flt
 \end{verbatim}
\end{screen}
at the shell prompt .  If you want to save the output it is easy to simply 
re-direct standard output to a file by doing 
\begin{screen}
 \begin{verbatim}
% felt foo.flt > foo.results
 \end{verbatim}
\end{screen}
Th file {\tt foo.results} could then be printed using whatever printer 
facilities exist at your site ({\em lp}, {\em lpr}, {\em print}, 
etc.)  If you want to plot 
the structure, the command line might look like this: 
\begin{screen}
 \begin{verbatim}
% felt -graphics foo.graph foo.flt > foo.results
% myplotter foo.graph
 \end{verbatim}
\end{screen}
The file {\tt foo.graph} could then be used with a program like {\em gnuplot}, 
{\em xmgr} or {\em xgraph}
(here we used a mystical program called {\em myplotter}) to visualize the 
structure and make sure that all the elements were connected properly.

\section{Interpreting the output from {\em felt}}
\label{felt_prog.output}

The output from {\em felt} is the most basic form of output that all 
applications in the \felt{} system produce.  That is, no matter how you specify
and solve a \felt{} problem (either with {\em felt} or 
{\em velvet}) the basic output will look the same.  

The main output will vary depending on the analysis type defined in the input
file.  In addition to the results from the specific analysis,
supplemental kinds of output which are available include material usage 
statistics, debugging information, and print-outs of the global stiffness and 
mass matrices.  Supplemental output from {\em felt} is controlled
via command line switches (see section~\ref{felt_prog.using}).

\subsection{Static analysis}

For a static analysis problem there are three sections of 
output: nodal displacements, element stresses and reaction forces.  
The nodal displacements are given in a table with the six global DOF
running across the top and the list of nodes going down.  The displacement 
of each node in each DOF is printed.  Of course in most problems not all 
DOF will be active so many of the displacements will simply come up as zero.

The information in the element stress table will vary depending on the 
element type.  Each row contains up to six columns of information; if an 
element calculates more than six stresses or loads, the information for that 
element will take up more than one row (beam3d elements for instance have 
12 loads calculated and thus the information for each element takes up two 
rows).  The output format for each element type is summarized in 
Table~\ref{felt_prog.stress_table}.

\begin{table}[tbh]
\begin{center}
 \begin{tabular}{|l|c|c|c|c|c|c|}
\hline
Element		& col 1	& col 2	& col 3	& col 4	& col 5	& col 6 \\
\hline\hline
truss		& $\sigma_x$ & & & & & \\
beam		& $f_x^1$ & $f_y^1$ & $m_z^1$ & $f_x^2$ & $f_y^2$ & $m_z^2$ \\ 
beam3d		& $f_x^1$ & $f_y^1$ & $f_z^1$ & $m_x^1$ & $m_y^1$ & $m_z^1$ \\ 
      	 	& $f_x^2$ & $f_y^2$ & $f_z^2$ & $m_x^2$ & $m_y^2$ & $m_z^2$ \\ 
timoshenko	& $f_y^1$ & $m_z^1$ & $f_y^2$ & $m_z^2$ & & \\ 
CSTPlaneStress	& $\sigma_x$ & $\sigma_y$ & $\tau_{xy}$ & $\sigma_1$ & $\sigma_2$ & $\theta$ \\
CSTPlaneStrain	& $\sigma_x$ & $\sigma_y$ & $\tau_{xy}$ & $\sigma_1$ & $\sigma_2$ & $\theta$ \\
quad\_PlaneStress & $\sigma_x^1$ & $\sigma_y^1$ & $\tau_{xy}^1$ & $\sigma_1^1$ & $\sigma_2^1$ & $\theta^1$ \\
                  & \vdots       & \vdots       & \vdots        & \vdots       & \vdots       & \vdots     \\
                  & $\sigma_x^4$ & $\sigma_y^4$ & $\tau_{xy}^4$ & $\sigma_1^4$ & $\sigma_2^4$ & $\theta^4$ \\
quad\_PlaneStrain & $\sigma_x^1$ & $\sigma_y^1$ & $\tau_{xy}^1$ & $\sigma_1^1$ & $\sigma_2^1$ & $\theta^1$ \\
                  & \vdots       & \vdots       & \vdots        & \vdots       & \vdots       & \vdots     \\
                  & $\sigma_x^4$ & $\sigma_y^4$ & $\tau_{xy}^4$ & $\sigma_1^4$ & $\sigma_2^4$ & $\theta^4$ \\
htk 		& $m_{xx}$ & $m_{yy}$ & $m_{xy}$ & $q_x$ & $q_y$ & \\
brick  		& $\sigma_{xx}^1$ & $\sigma_{yy}^1$ & $\sigma_{zz}^1$ & $\tau_{xy}^1$ & $\tau_{yz}^1$ & $\tau_{zx}^1$ \\
     		& \vdots          & \vdots          & \vdots          & \vdots        & \vdots        & \vdots        \\
     		& $\sigma_{xx}^8$ & $\sigma_{yy}^8$ & $\sigma_{zz}^8$ & $\tau_{xy}^8$ & $\tau_{yz}^8$ & $\tau_{zx}^8$ \\
  \hline
 \end{tabular}
\end{center}
\caption{Contents of the stress vector for different element types.}
\label{felt_prog.stress_table}
\end{table}

The reaction force section contains the global forces calculated at each 
constrained DOF.  The first column indicates the global node number at which 
this reaction occurs, the second column contains the DOF at which this force 
is applied and the 
last column gives the magnitude of this reaction force.

If we were to save the input file from the detailed example of 
section~\ref{problem.example1} to a file {\tt mixed.flt} then the command
\begin{screen}
 \begin{verbatim}
% felt -summary mixed.flt > mixed.result
 \end{verbatim}
\end{screen}
would produce a file, {\tt mixed.result}, which looked like the following.
\begin{screen}
 \begin{verbatim}
** Mixed Element Sample **

Nodal Displacements
-----------------------------------------------------------------------------
Node #      DOF 1       DOF 2       DOF 3       DOF 4       DOF 5       DOF 6
-----------------------------------------------------------------------------
  1             0           0           0           0           0  -0.0032019
  2             0   -0.011522           0           0           0           0
  3             0           0           0           0           0   0.0032019
  4             0           0           0           0           0           0

Element Stresses
----------------------------------------------------------------------------
  1:            0      29942           0          0      57.587           0
  2:            0     57.587      -59654          0       29942           0
  3:  -2.4196e+08

Reaction Forces
-----------------------------------
Node #     DOF     Reaction Force
-----------------------------------
  1        Tx                 0
  1        Ty             29942
  1        Tz                 0
  2        Tz                 0
  3        Ty             29942
  3        Tz                 0
  4        Tx                 0
  4        Ty            115.17
  4        Tz                 0
  4        Mz                 0

Material Usage Summary
--------------------------
Material: steel
Number:   2
Length:    12.0000
Weight:     0.0000

Material: spring
Number:   1
Length:    10.0000
Mass:      0.0000

Total mass:     0.0000
 \end{verbatim}
\end{screen}

\subsection{Transient analysis}

The output for transient analysis is much simpler.  It can consist of 
a table listing times and nodal displacements for all the nodes
and DOF which were specified in the \mbox{{\tt analysis parameters}}
section (or multiple tables if everything will not comfortably fit
across one screen width) and/or a time-displacement ASCII character
plot relating the same information in a graphical sense (or mulitple
plots if there are more than ten nodes/DOF).  Tabular output is the
default.  You can disable it with the command line switch {\tt +table}.
To see graphical output use {\tt -plot}.  

If we were to save the input file from the detailed example of 
section~\ref{problem.transient} to a file {\tt transient.flt} then the command
\begin{screen}
 \begin{verbatim}
% felt -plot transient.flt > transient.result
 \end{verbatim}
\end{screen}
would produce a file, {\tt transient.result}, which looked like the following.
{\small
\begin{screen}
 \begin{verbatim}
------------------------------------------------------------------
       time        Tx(8)
------------------------------------------------------------------
          0            0
       0.05   0.00077749
        0.1     0.028323
       0.15      0.10137
        0.2      0.23788
       0.25      0.45923
        0.3      0.76125
       0.35       1.1366
        0.4       1.5518
       0.45       1.9716
        0.5        2.352
       0.55       2.6652
        0.6       2.8977
       0.65       3.0442
        0.7       3.1074
       0.75         3.08
        0.8       2.9652

        0                            1.5537                           3.1074
        +------------------------------+-----------------------------------+
        |                                                                   
      0 x                                                                   
        |                                                                   
   0.05 x                                                                   
        |                                                                   
    0.1 x                                                                   
        |                                                                   
   0.15 | x                                                                 
        |                                                                   
    0.2 |    x                                                              
        |                                                                   
   0.25 |        x                                                          
        |                                                                   
    0.3 |               x                                                   
        |                                                                   
   0.35 |                       x                                           
        |                                                                   
    0.4 |                                x                                  
        |                                                                   
   0.45 |                                         x                         
        |                                                                   
    0.5 |                                                 x                 
        |                                                                   
   0.55 |                                                        x          
        |                                                                   
    0.6 |                                                             x     
        |                                                                   
   0.65 |                                                                x  
        |                                                                   
    0.7 |                                                                 x 
        |                                                                   
   0.75 |                                                                 x 
        |                                                                   
    0.8 |                                                              x    
        |                                                                   

      x x x    Tx(8)
 \end{verbatim}
\end{screen}}

\subsection{Modal analysis}

The output from a modal analysis of a simple one story frame is shown
below.  Note that the {\tt -matrices} command-line
switch was used to generate supplemental output (the matrices prints
before the results from the actual analysis) and that the {\tt -eigen}
switch was not activated so full modal analysis was performed.  
The actual modal analysis results consist of a listing of modal
frequencies, a table of mode shape vectors 
(the eigenvectors)\footnote{The current version of {\em felt} cannot 
generate a graphical representation of the mode shapes.  See the
discussion of mode shape plotting in chapter~\ref{velvet.solve} for
an example of how {\em velvet} can generate these kinds of plots.},
the modal mass, damping, and stiffness matrices and the damping ratios
in each mode.
\begin{screen}
 \begin{verbatim}
M = 

     15.62          0          0          0          0          0 
         0      15.62          0          0          0          0 
         0          0  8.558e+04          0          0          0 
         0          0          0      15.62          0          0 
         0          0          0          0      15.62          0 
         0          0          0          0          0  8.558e+04 

C = 

     194.8          0        720     -184.6          0          0 
         0      801.7      170.4          0     -1.049      170.4 
       720      170.4  1.123e+05          0     -170.4  1.846e+04 
    -184.6          0          0      194.8          0        720 
         0     -1.049     -170.4          0      801.7     -170.4 
         0      170.4  1.846e+04        720     -170.4  1.123e+05 

K = 

 9.711e+04          0    3.6e+05 -9.231e+04          0          0 
         0  4.005e+05  8.521e+04          0     -524.4  8.521e+04 
   3.6e+05  8.521e+04  5.446e+07          0 -8.521e+04  9.231e+06 
-9.231e+04          0          0  9.711e+04          0    3.6e+05 
         0     -524.4 -8.521e+04          0  4.005e+05 -8.521e+04 
         0  8.521e+04  9.231e+06    3.6e+05 -8.521e+04  5.446e+07 

** Modal Analysis Example **

Modal frequencies (rad/sec)
------------------------
Mode #      Frequency
------------------------
  1        12.049  (     1.9177 Hz)
  2        22.807  (     3.6299 Hz)
  3         30.09  (     4.7889 Hz)
  4        110.14  (     17.529 Hz)
  5           160  (     25.465 Hz)
  6        160.21  (     25.499 Hz)

Mode shapes
--------------------------------------------------------------------------
   Mode   1    Mode   2    Mode   3    Mode   4    Mode   5    Mode   6    
--------------------------------------------------------------------------
          1           1           1           1           1           1 
   0.003005  7.4866e-16   -0.011436 -6.6597e-17 -1.9981e+16       13775 
  -0.007032    -0.50358    0.025963  0.00036256   -0.012534      1.1007 
          1          -1           1          -1       3.667           1 
  -0.003005  -4.565e-15    0.011436  1.0303e-16 -1.9981e+16      -13775 
  -0.007032     0.50358    0.025963 -0.00036256   -0.041164      1.1007 

modal M = 

     39.71          0          0          0          0          0 
         0  4.344e+04          0          0          0          0 
         0          0      146.6          0          0          0 
         0          0          0      31.27          0          0 
         0          0          0          0  1.248e+34          0 
         0          0          0          0          0   5.93e+09 

modal K = 

      5766          0          0          0          0          0 
         0  2.259e+07          0          0          0          0 
         0          0  1.328e+05          0          0          0 
         0          0          0  3.794e+05          0          0 
         0          0          0          0  3.194e+38          0 
         0          0          0          0          0  1.522e+14 

modal C = 

     13.12          0          0          0          0          0 
         0  4.693e+04          0          0          0          0 
         0          0      271.4          0          0          0 
         0          0          0        760          0          0 
         0          0          0          0  6.393e+35          0 
         0          0          0          0          0  3.047e+11 

-------------------------
Modal damping ratios
-------------------------
  1         0.01371
  2         0.02368
  3         0.03075
  4         0.11032
  5         0.16013
  6         0.16034
 \end{verbatim}
\end{screen}

\subsection{Thermal analysis}

The result from a static thermal analysis is a simple table showing the
node number and corresponding steady state temperature value; thermal
stresses are not computed.

Output for transient thermal analysis can be tabular and/or graphical.
The tabulated results simply show the temperature as a function of time
for whatever nodes were listed in the {\tt nodes=} specification of the
analysis parameters (no {\tt dofs=} specification is necessary in 
transient thermal analysis).  If you specify the {\tt -plot} option on the
command line then {\em felt} will generate a simple ASCII plot of
time versus temperature for these same nodes of interest.  Just as
with transient structural analysis, tabular output can be turned off
witht the {\tt +table} command-line switch.

\subsection{Spectral analysis}

Depending on how {\em felt} is invoked and the nature of the problem, the
output for a spectral analysis problem can either consist of transfer
functions or of output spectra.  If the {\tt -transfer} switch is given
on the command line then all output will be for transfer functions only.
The transfer functions computed will be for all combinations of output at the 
nodes and DOF defined by the {\tt nodes=} and {\tt dofs=} statements in the
{\tt analysis parameters} section of the input file and input at all DOF
with a force applied.  For example, if you define nodes 2 and 4
and DOF Tx and Ty as the relevant DOF in the analysis parameters and
there is force object with Fx and Fy non-zero applied to both nodes 3 and 7,
then you will get sixteen transfer functions as results.  If the input
forcing is defined in a spectral sense (e.g., with {\tt Sfx=}) then
you can also compute the output spectra at the output DOF due to input
at any forced DOF with spectral inputs.  This is the default behavior when
the {\tt -transfer} switch is not specified on the command-line.  If none
of the forces in a problem have non-zero spectral input components then
the only solution that is non-trivial is with the {\tt -transfer} switch
invoked.

Just like for transient analysis, the output for spectral analysis (either
transfer functions or output spectra) can be tabular and/or plotted.  
ASCII plots of transfer functions or output spectra (whichever are appropriate
for the current problem) are generated if the {\tt -plot} switch is specified.
Tabular output is generated by default and can be turned off with the
{\tt +table} switch.
