%%%%%%%%%%%%%%%%%%%%%%%%%%%%%%%%%%%%%%%%%%%%%%%%%%%%%%%%%%%%%%%%%
%                                                               %
%                       Legal Notice                            %
%                                                               %
% This document is copyright (C) Jason Gobat & Darren Atkinson	%
%                                                               %
%%%%%%%%%%%%%%%%%%%%%%%%%%%%%%%%%%%%%%%%%%%%%%%%%%%%%%%%%%%%%%%%%

\newpage{\pagestyle{empty}\cleardoublepage}

\chapter{Introduction to the \felt{} System}
\label{intro}

\section{Intentions}
\label{intro.intentions}
%%%%%%%%%%%%%%%%%%%%%%%%%%%%%%%%%%%%%%%%%%%%%%%%%%%%%%%%%%%%%%%%%%%%%%%%%%%
\felt{} is a package for introductory level Finite ELemenT analysis.  It is
centered around a mathematical engine designed to simply, effectively,
accurately, and flexibly solve most types of structural / mechanical problems
that would be encountered in an introductory course in finite element
analysis.  It was developed in an overzealous fit of ``we can do better
than that'' based on some antiquated Fortran code that we had been using for 
just such a course. 

Our intention was to design a system capable of doing everything that 
code could do in the mathematical sense, but bring it into the 90's (or at 
least the late 80's) in terms of input syntax (including error checking, 
problem debugging, etc.), flexibility, and most importantly, a graphical user
interface.  We hope we've succeeded, or at the least, are on the right
track; \felt{} is a work-in-progress and chances are that if a 
feature isn't there now, there are at least pipe dreams of it somewhere
in the backs of our minds.   

Of course all our good intentions are for nought if no one uses \felt{}.  If
you find something wrong, let us know (our email addresses are in the
legal notice).  Alternatively (and an even better option) you can subscribe to
the \felt{} mailing list by sending a one line email message that says
``subscribe felt-l'' (without the quotes) to listserv@mecheng.fullfeed.com.
If something is not there that you think should be, let us know.
We make no promises, but we do try to respond and gear development toward 
user feedback.  If you're happily using \felt{} and have no complaints, let 
us know that too.  At least then we know that we're on the right track.

\section{\felt{}: What it can do for you}
%%%%%%%%%%%%%%%%%%%%%%%%%%%%%%%%%%%%%%%%%%%%%%%%%%%%%%%%%%%%%%%%%%%%%%%%%%%
\felt{} currently has support for the following element types:
three-dimensional truss or bar, one-dimensional spring, two- and 
three-dimensional Euler 
beam, two-dimensional Timoshenko beam, constant strain
triangles (CSTs) for both plane stress and plane strain analysis, planar 
isoparametric elements (four to nine node and a separate type for simple
four node quadrilaterals), again for both plane stress or plane strain
analysis, a linear axisymmetric triangular element, an HTK plate bending 
element, an isoparametric eight-node brick
element, and a rod and a constant temperature gradient triangle for thermal
analysis.  \felt{} allows for an arbitrary mixing of element types within a
problem.  The syntax
for the \felt{} input file is based on a high-level grammar which frees 
you from comma-delimited lists of numbers and hours of debugging due to
not having the right number after the right comma; both the parser and the
mathematical routines do extensive error checking and report what we hope
are informative and useful error messages.

\felt{} offers several user interface options.  The basic {\em felt}
application gives you the capability to define an input file in a
favorite editor and then solve the problem from the shell command line.
Graphics for the {\em felt} applications can be handled by any number of
graphing packages.   Under Windows, {\em WinFElt} provides a text
editor and encapsulator environment with some post-processing capability.
{\em velvet} is the full-featured graphical user interface to the 
\felt{} system.  {\em velvet} (which is
smoother than {\em felt}) knows most of what there is to know about the
\felt{} system and provides a consistent, CAD-like interface for drawing,
defining, solving, and visualizing everything about a problem.  Though it's
probably not always the best way to set-up a problem (it's hard to beat
{\em vi} for quick-and-dirty problems), a user working strictly with 
two-dimensional problems (three-dimensional graphics are only partially
supported in {\em velvet}) need never actually see the internals of a
\felt{} file or run the {\em felt} application from the shell command line;
{\em velvet} provides access to all of the two-dimensional functionality of
\felt{} in one completely stand-alone application.  {\em velvet's} current
post-processing capabilities include plotting the displaced shape,
two-dimensional color contours of stress and displacement for planar elements,
animation of dynamic structural simulations, line plots of for 
time and frequency domain results, and graphical presentation of mode shapes.

Automated element generation for a \felt{} problem is provided for both
simple grids of line, quadrilateral and brick elements and arbitrary meshes
of triangular planar elements.  The latter capability is derived
from J.R. Shewchuk's Triangle mesh generation routine.
You can interface this functionality either graphically through {\em velvet}
or through a separate command line application called {\em corduroy} that
has its own input file syntax much like the regular syntax for \felt{}
problems.

Support applications are provided for file format conversion and unit 
conversion and problem scaling.  {\em patchwork} can translate between
the standard \felt{} syntax and several other common graphical description
formats.  {\em yardstick} can be used to scale numerical quantities within
a \felt{} file, including special options for conversion between different
types of units.

\felt{} should be able to handle most types of linear static and dynamic 
problems that
you throw at it, but there are no guarantees.  Most elements allow
arbitrary oriented distributed line loads.  Displacement (e.g., settlement of 
support) and force (e.g., nodal hinge) boundary conditions are also
allowed.  Time varying force and boundary conditions can be expressed either
as continuous or discrete functions.

\section{\felt{}: What it cannot do for you}
%%%%%%%%%%%%%%%%%%%%%%%%%%%%%%%%%%%%%%%%%%%%%%%%%%%%%%%%%%%%%%%%%%%%%%%%%%%
As of this release, \felt{} can only handle linear static and dynamic problems.  
We realize the shortcomings that this presents for some people and we have 
some vague plans for non-linear analysis, but nothing is here yet.
{\em velvet} can't really draw in 3-d (at least in the problem definition 
stage) and thus isn't a terribly good way to define 3-d problems; it will 
always assume that it should work in the x-y plane (z = 0).  This is probably 
going to stay this way for a long time.  

There are certainly other shortcoming as well, depending on just what you 
would like the package to do.  What it really comes down to is that \felt{}
was never intended to solve everybody's real-world or cutting-edge research
problems, so we're probably never going to incorporate lots of different
analysis types, etc.  If you want to take a crack at modifying \felt{} for your
own local needs, however,  then we encourage you to do so; we'll even help 
out where possible.
